\chapter{Introdução}

    A empresa 
    Rio Branco Alimentos SA/Fábrica de Rações de Patrocínio
    enfrenta um problema relacionado a falta de monitoramento 
    adequado da temperatura dos freezeres que guardam as vacinas
    da empresa.

    \begin{citacao}
        Atualmente temos 5 unidades de conservadoras de vacinas contendo 
        em média 30.000 doses cada uma.
        No passado tivemos problemas com perca de vacinas devido 
        defeito em uma das conservadoras levando a um prejuízo financeiro.
        Hoje não temos sistema de monitoramento automatizado ou meios 
        que garantam a conservação da vacina com a 
        manutenção da temperatura ideal. \cite{senaiDemanda}
    \end{citacao}

    Com o propósito de sanar esse problema decidiu-se por desenvolver
    um sistema para o gerenciamento da temperatura dos freezeres
    de cada unidade, de forma a aumentar a qualidade dos produtos
    entregues aos consumidores.

    Em relação a parte física do projeto consiste basicamente
    em instalar sensores de temperatura em cada freezer e 
    também instalar um sensor na porta para detectar se a 
    mesma esta aberta ou fechada. E cada sensor deverá transmitir 
    em tempo real os dados para um servidor e dessa forma 
    criar um histórico de variação de temperatura podendo 
    ser acessível quando necessário.

    Já sobre a parte do software o mesmo deve ter uma tela para
    poder cadastrar, atualizar e excluir os sensores do sistema, 
    e outra para visualização do histórico de temperatura e 
    e das variações em tempo real, mostrando se a porta esta 
    aberta ou fechada e emitindo um alerta caso a temperatura
    saia do nível especificado.

    A seleção da demanda foi feita através do portal Saga Senai, um 
    site onde empresas podem cadastrar seus problemas ou 
    dificuldades e os alunos do SENAI podem tentar resolver 
    esses problemas de forma inovadora. 
    Com isso os autores desse trabalho da turma TII2004M
    do curso de Informática para Internet do segundo modulo
    decidiram se desafiar a idealizar 
    uma solução estratégica inovadora para a situação 
    proposta pela empresa 
    Rio Branco Alimentos SA/Fábrica de Rações de Patrocínio.

    % Para o desenvolvimento desse projeto foi escolhido no 
    % Portal Saga Senai a demanda de
    % \href{http://plataforma.gpinovacao.senai.br/plataforma/demandas-da-industria/interna/4804}{monitoramento e conservação de vacinas}
    % onde a empresa tem uma dificuldade em monitorar a temperatura 
    % dos freezeres, para poder manter as vacinas em bom estado 
    % e assim evitar possíveis perdas.
